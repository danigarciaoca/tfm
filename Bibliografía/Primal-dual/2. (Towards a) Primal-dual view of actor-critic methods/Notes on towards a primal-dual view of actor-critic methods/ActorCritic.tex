\documentclass[11pt,onecolumn,draftcls]{IEEEtran}
%\documentclass[journal]{./IEEEtran}

% TTT IEEEtran.cls PACKAGES TTT
\usepackage{ifpdf}
\usepackage{cite}
\ifCLASSINFOpdf
 \usepackage[pdftex]{graphicx}
 \DeclareGraphicsExtensions{.pdf,.jpeg,.png}
\else
 \usepackage[dvips]{graphicx}
 \DeclareGraphicsExtensions{.eps}
\fi
\usepackage{amssymb}
\usepackage[cmex10]{amsmath}
\interdisplaylinepenalty=2500
\usepackage{array}
\usepackage{mdwmath}
\usepackage{mdwtab}
\usepackage{eqparbox}
\usepackage[tight,footnotesize]{subfigure}
\usepackage{stfloats}

\newcommand{\hilight}[1]{\colorbox{yellow}{#1}}

%---------------------------------------------------------------
% Packages added by the author
%---------------------------------------------------------------
\usepackage{bm}
\usepackage{overpic}
\usepackage{color}
\usepackage{dsfont}
\usepackage{amsthm}
\usepackage{setspace}
\usepackage{mathbbol}
%\usepackage{algorithmicx}
%\usepackage{algpseudocode}
%\usepackage[ruled]{algorithm}
%\usepackage[named]{algo}
\usepackage[ruled,norelsize]{algorithm2e}
\usepackage{bbm}
\usepackage{boxedminipage}

%\IEEEeqnarraydefcolsep{0}{\leftmargini}


%---------------------------------------------------------------
% Also added by the author
%---------------------------------------------------------------
\newcommand{\E}{\mathsf{E}}
\newcommand{\T}{\top}
\newcommand{\mc}{\mathcal}
\newcommand{\defeq}{\triangleq}

\newcommand{\1}{\bm{1}}
\newcommand{\0}{\boldsymbol{0}}
\newcommand{\myr}{\mathbf{r}}
\newcommand{\myv}{\mathbf{v}}
\newcommand{\mymu}{\boldsymbol{\mu}}
\newcommand{\myd}{\mathbf{d}}
\newcommand{\myh}{\mathbf{h}}
\newcommand{\myg}{\mathbf{g}}




%\renewcommand{\Re}{\mathbb{R}}
\newcommand{\St}{\mathbb{S}}
\newcommand{\Ac}{\mathbb{A}}
\newcommand{\Na}{\mathbb{N}}
\newcommand{\Xf}{\mathbb{X}}

\newcommand{\diag}{\mathrm{diag}}
\newcommand{\N}{\mc{N}}
\newcommand{\col}{\mathrm{col}}
\newcommand{\rank}{\mathrm{rank}}

\newcommand{\Pb}{\mathsf{P}}

\newcommand{\vm}{\vspace{5px}}

\newtheorem{assumption}{Assumption}
\newtheorem{lemma}{Lemma}
\newtheorem{theorem}{Theorem}
\newtheorem{corollary}{Corollary}
\newtheorem{remark}{Remark}






%\allowdisplaybreaks 

% Title.
% ------
\title{(Towards a)\\ Primal-dual view of actor-critic methods}

\begin{document}

\author{
%Sergio~Valcarcel~Macua,~\IEEEmembership{Student Member,~IEEE}
}


%\markboth{IEEE TRANSACTIONS ON SIGNAL PROCESSING,~Vol.~XX, No.~XX, Month~2013}%
%{Macua \MakeLowercase{\textit{et al.}}: Bare Demo of IEEEtran.cls for Journals}

\maketitle

%\begin{abstract}
%\end{abstract}
%

%\begin{IEEEkeywords}
%Actor-critic,
%gradient temporal difference,
%mean-square-error,
%reinforcement learning, 
%stochastic approximation
%\end{IEEEkeywords}



% For peer review papers, you can put extra information on the cover
% page as needed:
\ifCLASSOPTIONpeerreview
\begin{center} 
	\bfseries EDICS Categories: 
	MLR-DIST; SEN-DCON; SEN-COLB; MLR-COGP 
\end{center}
\fi
%
% For peerreview papers, this IEEEtran command inserts a page break and
% creates the second title. It will be ignored for other modes.
\IEEEpeerreviewmaketitle


% =====================================
%
\section{Primal-Dual Linear Programming Formulation of MDP}
\label{sec:primal-dual}
%
% =====================================

%\IEEEPARstart{F}{irst}

Let $S$ and $A$ denote state and action sets, respectively.
%
Define the transition matrix $P$, of size $|S||A| \times |S|$, 
whose entries $P_{(sa,s')}$ specify the conditional probability of going to state $s'$ when starting from state $s$ and taking action $a$: 
%
\begin{IEEEeqnarray}{rCl}
P_{(sa,s)} = p (s'|s,a) \ge 0
,\quad
\sum_{s' \in S} p (s'|s,a) = 1 
\end{IEEEeqnarray}
%
Let $r(s,a,s')$ denote the reward obtained when taking action $a$ in state $s$ and transitioning to state $s'$.
We also define $r(s,a)$ as the reward obtained when taking action $a$ in state $s$, such that
\begin{IEEEeqnarray}{rCl}
r(s,a) = \sum_{s'} p (s'|s,a) r(s,a,s')
\end{IEEEeqnarray}
%
Let $\pi$ denote the policy, 
such that $\pi(a, s) = p(a |s)$ is the probability of taking action $a$ in state $s$.



We can consider both average and discounted formulations.
For simplicity, we focus on discounted formulation with bounded rewards.
%
%
% ------------------------------------------
\begin{assumption}
%
There exists some scalar $M$ such that the reward $r$ satisfies
\begin{IEEEeqnarray}{rCl}
	| r(s,a) |  \le M
,\quad
	\forall	(s,a) \in S \times A
\end{IEEEeqnarray}
%
\end{assumption}
% ------------------------------------------
%
Define the value function as the long-term expected discounted cumulative reward:
\begin{IEEEeqnarray}{rCl}
v^{\pi}(s) 
	&
	=
	& 
		\E 
		\left \{
				\sum_{i=0}^{\infty}	\gamma^{i} r (s_i, a_i)		
			\: | \:
				a_i \sim \pi(\cdot| s_i)
			,\;
				s_0 = s
		\right \}
\end{IEEEeqnarray}
%
where $0 < \gamma < 1$ is the discount factor.
%
Introduce the optimal value function
\begin{IEEEeqnarray}{rCl}
v^{\star} (s) 
\defeq
	\max_{\pi \in \Pi} 
		v^{\pi}(s)
,\qquad
	\forall s \in S
.
\label{eq:optimal-value-function}
\end{IEEEeqnarray}
%
%
%
Our goal is to find the policy that maximizes the long-term reward:
%
\begin{IEEEeqnarray}{rCl}
\pi^{\star}(s)
\defeq
	\arg \max_{a \in \Ac} 
		\left( 
			r(s,a) + \gamma \sum_{s' \in \St} P(s'|s,a) v^{\star}(s')
		\right)
,\quad
	\forall s \in S
\label{eq:optimal action}
\end{IEEEeqnarray}

%
Introduce the dynamic programming operator $T$:
%$T^{\pi}$ and $T$:
%
\begin{IEEEeqnarray}{rCl}
%T^{\pi} v 
%&
%\defeq
%&
%	r^{\pi} + \gamma P^{\pi} v
%
%\\
(T v) (s)
&
\defeq
&
	\max_{a \in A}
		\left(
			r(s,a) + \gamma \sum_{s' \in \St} P(s'|s,a) v (s')
		\right)	
\label{eq:dp-operator}
\end{IEEEeqnarray}
%
for which the following results are well known.
%
% ------------------------------------------
\begin{lemma}
\label{lemma:dp-operator}
%
The dynamic programming operator $T$ satisfies the following properties:
\begin{enumerate}
%
	\item {\rm (Monotonicity)}	
		For any functions 
		$v : S \rightarrow \Re$
		and
		$v' : S \rightarrow \Re$, 
		such that
		\begin{IEEEeqnarray}{rCl}
		 v(s) \le v' (s)
		 ,\; \forall s \in S
		\end{IEEEeqnarray}
		we have 
		\begin{IEEEeqnarray}{rCl}
		 (T^i v) (s) \le (T^i v') (s)
		 ,\quad
		 \forall s\in S
		 ,\; i=1, 2, \ldots
		\end{IEEEeqnarray}
		%
		where 
		$
		(T^i v)(s) 
		= (T (T^{i-1} v))(s) 
		= (T (T \cdots T v)) (s)$.
%
	\item {\rm (Convergence)} 
		For any bounded function $v : S \rightarrow \Re$,
		the optimal value function satisfies
		\begin{IEEEeqnarray}{rCl}
		v^{\star}(s) = \lim_{i \rightarrow \infty} (T^i v) (s)
		,\quad
		\forall s \in S
		\end{IEEEeqnarray}
%
	\item {\rm (Bellman's equation)} 
		The optimal value function satisfies 
		and is the unique solution to
		\begin{IEEEeqnarray}{rCl}
		 v^{\star} (s) = (T v^{\star}) (s)
		,\quad
		\forall s \in S
		\end{IEEEeqnarray}
%		
\end{enumerate}
%
\end{lemma}
%
\begin{IEEEproof}
See, e.g., \cite{bertsekas2012dynamic}.
\end{IEEEproof}
% ------------------------------------------
%


Let us write $v' \ge v$ to denote that $v'(s) \ge v(s)$, for all $s$.
%
From Lemma \ref{lemma:dp-operator}.3, 
we have 
%
\begin{IEEEeqnarray}{rCl}
	v \ge T v 		\quad \Rightarrow \quad v \ge v^\star = T v^\star
\end{IEEEeqnarray}
%
Thus, $v^\star$ is the smallest $v$ that satisfies the constraint $v \ge TJ$.
%
This constraint can be written as a finite set of linear inequalities
%
\begin{IEEEeqnarray}{rCl}
	v(s) 	\ge 		r(s,a) + \gamma \sum_{s' \in \St} p(s'|s,a) v(s'), 
	\quad 
	\forall s \in S
	,\;
	\forall a \in A
\end{IEEEeqnarray}
%
which delineates a polyhedron in $\Re^{|S||A|}$.
%
The optimal value is the "shoutheast" corner of this polyhedron.
%

%
Write the value in compact vector form:
\begin{IEEEeqnarray}{rCl}
\myv^{\pi} \defeq 
	\left( 
		v^{\pi}(s) 
	\right)_{s \in S}
\end{IEEEeqnarray}
%
Hence, the optimal value vector is given by
%
\begin{IEEEeqnarray}{rCl}
\myv^{\star}
\defeq
	\max_{\pi \in \Pi} 
		\myv^{\pi}
\label{eq:optimal-value-vector}
\end{IEEEeqnarray}
%
We can find $\myv^\star$ by solving the following linear programming problem
\cite{bertsekas2012dynamic,deFarias2003linear,wang2008dual}:
\begin{IEEEeqnarray}{rCl}
\mc{P}_{0}:
\quad
	\begin{split}
		\underset{\myv}{\rm minimize} 	
			&\quad 		
				\1^{\T} \myv 
\\
		{\rm s.t.} 					
			&\quad 	
				v(s) \ge 	r(s,a) + \gamma \sum_{s' \in \St} p(s'|s,a) v(s'), 
				\quad 
					\forall s \in S
				,\;
					\forall a \in A
	\end{split}
\label{eq:linear-programming-problem}
\end{IEEEeqnarray}
%
Instead of minimizing $\1^{\T} \myv$, 
let us minimize the objective $(1-\gamma) \mymu^{\T} \myv$, 
where 
$0< \gamma < 1$
and $\mymu$ can be seen as the probability distribution over the initial state, 
given that $\mymu \ge 0$ and $\1^{\T} \mymu = 1$.
Hence, the primal problem becomes
%
\begin{IEEEeqnarray}{rCl}
\mc{P}_1:
\quad
	\begin{split}
		\underset{\myv}{\rm minimize} 	
			&\quad 		
				(1-\gamma)\mymu^{\T} \myv 
\\
		{\rm s.t.} 					
			&\quad 	
				v(s) \ge 	r(s,a) + \gamma \sum_{s' \in \St} p(s'|s,a) v(s'), 
				\quad 
					\forall s \in S
				,\;
					\forall a \in A
	\end{split}
\label{eq:alternative-linear-programming-problem}
\end{IEEEeqnarray}
%
Although this change has no influence on problem \eqref{eq:linear-programming-problem}
(i.e., it has the same solution as \eqref{eq:alternative-linear-programming-problem}),
reference \cite{wang2008dual} shows that this proportional factor plays an important non-arbitrary role when we formulate its dual problem.
In particular, $\mymu$ can be seen as a probability distribution over the initial state.
%
Introduce the the reward vector
%
\begin{IEEEeqnarray}{rCl}
\myr \defeq 
	\left( 
		r(s, a) 
	\right)_{s \in S, a\in A}
\in 
	\Re^{|S||A|}
\end{IEEEeqnarray}
%
and the marginalization matrix, $\Phi$, of size  $|S| \times |S||A|$,
such that we can express the inequality constraints in 
\eqref{eq:alternative-linear-programming-problem} as
%
\begin{IEEEeqnarray}{rCl}
\Phi^\T \myv
\ge
	r + \gamma P \myv
\end{IEEEeqnarray}
%
The Lagrangian of \eqref{eq:alternative-linear-programming-problem},
with dual variable $\myd \ge 0$,
is given by
%
\begin{IEEEeqnarray}{rCl}
L(\myv, \myd) 
= 
	(1-\gamma) \mymu^\T \myv
	+
	\myd^\T
	(\myr + \gamma P \myv - \Phi^\T \myv)
\label{eq:lagrangian-primal-linear}
\end{IEEEeqnarray}
%
The dual function is defined as
%
\begin{IEEEeqnarray}{rCl}
g(\myd) 
= 
	\inf_{\myv} L(\myv, \myd)
\end{IEEEeqnarray}
%
The first order condition for minimizing the Lagrangian over the primal variable  $\myv$ 
is given by:
%
\begin{IEEEeqnarray}{rCl}
\nabla_{\myv} L(\myv, \myd) 
= 
	(1-\gamma) \mymu
	+
	\gamma P^\T \myd 
	-
	\Phi \myd 
=
	\0
\end{IEEEeqnarray}
%
Hence,
%
\begin{IEEEeqnarray}{rCl}
g(\myd) 
=
\begin{cases}
	\myd^\T \myr		& 	{\rm if } \; \Phi \myd = (1-\gamma) \mymu + \gamma P^\T \myd \\
	-\infty			&		{\rm otherwise}
\end{cases}
\end{IEEEeqnarray}
%
Therefore, 
the dual problem is given by
%
\begin{IEEEeqnarray}{rCl}
\mc{P}_2:
\quad
	\begin{split}
		\underset{\myd}{\rm maximize} 	
			&\quad 		
				\myd^{\T} \myr
\\
		{\rm s.t.} 					
			&\quad 	
				\Phi \myd = (1-\gamma) \mymu + \gamma P^\T \myd
\\
			&\quad 	
				\myd \ge 	0
	\end{split}
\label{eq:dual-programming-problem}
\end{IEEEeqnarray}
%
%
Let $\myd^{\star}$ denote the solution to \eqref{eq:dual-programming-problem}.
%
By strong duality, 
we know that the optimal objective value of the dual equals the optimal value function:
%
\begin{IEEEeqnarray}{rCl}
(1-\gamma) \mymu^\T \myv^{\star} = \myd^{\star\T} \myr
\label{eq:strong-duality}
\end{IEEEeqnarray}
%
Before unrolling the left side of \eqref{eq:strong-duality},
we introduce the optimal discounted stationary distributions over states conditioned on the initial state:
%
\begin{IEEEeqnarray}{rCl}
\xi^{\star} (s,x) 
&=&
\sum_{i=0}^{\infty} \gamma^i p(s_i =s | s_0 = x, \: a_i \sim \pi^{\star})
\end{IEEEeqnarray}
%
Now, by using the following relationship due to \cite{sutton1999policy}:
%
\begin{IEEEeqnarray}{rCl}
v^{\star} (x) 
&=&
	\sum_{s \in S} 
		\xi^{\star} (s,x) 
		\sum_{a \in A}
			\pi^{\star}(s,a) 
			r(s,a)
\end{IEEEeqnarray}
%
we can unroll the left side of \eqref{eq:strong-duality} as follows:
%
\begin{IEEEeqnarray}{rCl}
(1-\gamma) \mymu^\T \myv^{\star} 
&=& 
	(1-\gamma) 
	\sum_{x \in S} 
		\mu(x)
		\sum_{s \in S}
			\xi^{\star} (s,x)
			\sum_{a \in A}
				\pi^{\star}(s,a) 
				r(s,a)
\notag \\
&=& 
	\sum_{s \in S}
		\mu^{\star} (s)
			\sum_{a \in A}
				\pi^{\star}(s,a) 
				r(s,a)
\label{eq:unroll-optimal-value}
\end{IEEEeqnarray}
%
where
%
\begin{IEEEeqnarray}{rCl}
\mu^{\star} (s)
&=& 
	(1-\gamma) 
	\sum_{x \in S}
		\mu (x)
		\xi^{\star}(s,x)
,\quad
	\mu^{\star} (s) \ge 0,
\;
	\forall s \in S
\quad
{\rm and}
\quad
	\sum_{s \in S}
		\mu^{\star} (s) = 1
\end{IEEEeqnarray}
%
%
%
In order to unroll the right side of \eqref{eq:strong-duality},
we write
%
\begin{IEEEeqnarray}{rCl}
\myd^{\star\T} \myr 
&=&
	\sum_{s \in S} 
		\sum_{a \in A}
			d^{\star}(s,a) r(s,a)
\notag \\
&=&
	\sum_{s \in S} 
		h^{\star}(s)
		\sum_{a \in A}
			\phi^{\star}(s,a) r(s,a)
\label{eq:dual-average-reward}
\end{IEEEeqnarray}
%
where 
%
\begin{IEEEeqnarray}{rCl}
h^{\star}(s) 
&=& 
	\sum_{a \in A} d^{\star}(s,a)
,\quad
\phi^{\star}(s,a) 
=
	\frac{d^{\star}(s,a)}{\sum_{a \in A} d^{\star}(s,a)}
\label{eq:terms-unroll-dual}
\end{IEEEeqnarray}
%
We know that $\myd^{\star} \ge 0$
and reference \cite[Lemma 1]{wang2008dual} shows that $\1^\T \myd^{\star} = 1$.
Hence, we can see $\myd^{\star}$ as a joint probability distribution over state-action pairs.
Moreover, the following properties establish that $h^{\star}$ and $\phi^{\star}$ 
are also probability distributions:
%
\begin{IEEEeqnarray}{rCl}
	h^{\star}(s) \ge 0
,\quad 
	\sum_{s \in S} h^{\star}(s) &=& 1
,\quad 
	\phi^{\star}(s,a) \ge 0
,\quad
	\sum_{a \in A} \phi^{\star}(s) = 1
\label{eq:terms-unroll-dual-probability}
\end{IEEEeqnarray}
%
%
From \eqref{eq:strong-duality},
\eqref{eq:unroll-optimal-value}
and \eqref{eq:dual-average-reward},
we have:
%
\begin{IEEEeqnarray}{rCl}
\sum_{s \in S}
	\mu^{\star} (s)
		\sum_{a \in A}
			\pi^{\star}(s,a) 
			r(s,a)
&=&
	\sum_{s \in S} 
		h^{\star}(s)
		\sum_{a \in A}
			\phi^{\star}(s,a) r(s,a)
\label{eq:unrolled-strong-duality}
\end{IEEEeqnarray}
%
From 
\eqref{eq:terms-unroll-dual},
\eqref{eq:terms-unroll-dual-probability}
and
\eqref{eq:unrolled-strong-duality},
we conclude that we can obtain the optimal policy from the optimal dual variable:
%
\begin{IEEEeqnarray}{rCl}
\pi^{\star}(s,a)
&=&
\frac{d^{\star}(s,a)}{\sum_{a \in A} d^{\star}(s,a)}
\label{eq:policy-from-optimal-dual-variable}
\end{IEEEeqnarray}
%
%
This conclusion is very interesting:
while the primal linear problem searches the optimal value, 
the dual formulation searches in the policy space.
% 
This suggests that we can develop a primal-dual method that 
finds a saddle point of the Lagrangaian 
by searching both in the value function and the policy spaces.




% =====================================
%
\section{Saddle-Point formulation of Actor-Critic Methods}
\label{sec:saddle-point}
%
% =====================================

If we derive the dual of the dual problem \eqref{eq:dual-programming-problem},
we recover the original primal problem \eqref{eq:alternative-linear-programming-problem}.
%
Indeed, 
since problems \eqref{eq:alternative-linear-programming-problem} and \eqref{eq:dual-programming-problem}
are convex and satisfy Slater’s condition \cite{boyd2004convex},
strong duality holds and their respective primal and dual optimal values
are attained and equal and they form a saddle-point of their Lagrangian.


Recall from \eqref{eq:lagrangian-primal-linear}
the Lagrangian of the primal problem \eqref{eq:alternative-linear-programming-problem}:
\begin{IEEEeqnarray}{rCl}
L_{\mc{P}_1}(\myv, \myd) 
= 
	(1-\gamma) \mymu^\T \myv
	+
	\myd^\T
	(\myr + \gamma P \myv - \Phi^\T \myv)
\label{eq:lagrangian-primal}
\end{IEEEeqnarray}
%
First order conditions for a saddle point of \eqref{eq:lagrangian-primal} are:
%
\begin{IEEEeqnarray}{rCl}
\nabla_{\myv} L_{\mc{P}_1}(\myv, \myd) 
&=& 
	(1-\gamma) \mymu
	+
	\left(
		\gamma P^\T - \Phi
	\right) 
	\myd
=
	\0
\label{eq:saddle-point-v-primal}
\\
\nabla_{\myd} L_{\mc{P}_1}(\myv, \myd) 
&=& 
	\myr + \gamma P \myv - \Phi^\T \myv
=
	\0
\label{eq:saddle-point-d-primal}
\end{IEEEeqnarray}

The Lagrangian of the dual problem \eqref{eq:dual-programming-problem} is given by:
%
\begin{IEEEeqnarray}{rCl}
L_{\mc{P}_2}(\myd, \myg, \myv) 
= 
	\myd^\T \myr 
	+ 
	\myg^\T \myd 
	+
	\myv^\T
	\left( 
		(1-\gamma) \mymu 
		+ 
		\gamma P^{\T} \myd 
		- \Phi \myd
	\right)
\label{eq:lagrangian-dual}
\end{IEEEeqnarray}
%
with first order conditions for a saddle point given by
%
\begin{IEEEeqnarray}{rCl}
\nabla_{\myd} L_{\mc{P}_2}(\myd, \myg, \myv) 
&=& 
	\myr 
	+ 
	\myg
	+
	\gamma P \myv
	-
	\Phi^{\T} \myv
=
	\0
\label{eq:saddle-point-d-dual}
\\
\nabla_{\myg} L_{\mc{P}_2}(\myd, \myg, \myv) 
&=& 
	\myd	
=
	\0
\label{eq:saddle-point-g-dual}
\\
\nabla_{\myv} L_{\mc{P}_2}(\myd, \myg, \myv) 
&=& 
	(1-\gamma) \mymu
	+
	\gamma P^\T \myd
	-
	\Phi \myd
=
	\0
\label{eq:saddle-point-v-dual}
\end{IEEEeqnarray}



\textbf{To do:} Now, the idea could be to derive standard and derive novel actor-critic methods from these first order conditions.



% =====================================
%
\section{Some Remarks to Explore}
\label{sec:remarks}
%
% =====================================


\begin{remark}
I guess that \eqref{eq:dual-average-reward} and \eqref{eq:terms-unroll-dual} 
are related to the deterministic policy gradient theorem \cite{silver2014deterministic}.
\end{remark}






\bibliographystyle{IEEEtran}
\bibliography{IEEEabrv,refs}








\end{document}
